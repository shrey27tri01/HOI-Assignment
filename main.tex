\documentclass[12pt]{article}

%% Language and font encodings
\usepackage[english]{babel}
%\usepackage[utf8x]{inputenc}
%\usepackage[T1]{fontenc}
\usepackage{fontspec}
\usepackage{url}
\setmainfont{Times New Roman}

%% Sets page size and margins
\usepackage[a4paper,top=3cm,bottom=3cm,left=3cm,right=3cm,marginparwidth=1.75cm]{geometry}

%% Useful packages
\usepackage{amsmath}
\usepackage{graphicx}
\usepackage[colorinlistoftodos]{todonotes}
\usepackage[colorlinks=true, allcolors=blue]{hyperref}

\title{HSS 102 History of Ideas: Take Home Assignment}
\author{IMT2019084 Shrey Tripathi}

\begin{document}
\maketitle

\textbf{Question:}\\
Which social contract theory (as espoused by Hobbes/Locke/Rousseau) bears
considerable resemblance to present State of India and Why? \\\\

\textbf{Answer:}\\
The social contract theories given by Thomas Hobbes, John Locke, and Jean-Jacques Rousseau are widely different and show how differently these philosophers perceived sovereignty, politics and human nature. 

Thomas Hobbes presented his ideas to the world in his book \textbf{Leviathan}. \textit{Leviathan} was a biblical sea monster from ancient times that was believed to be the most powerful creature on Earth. In this book, Hobbes very ingeniously took people back to ancient times, before there were kings of any kind, and argued that when humans are left to themselves without a central authority to keep them in awe, they tend towards a \textit{state of nature}, where they quickly descend into a squatting, infighting and an intolerable bickering mess. Since there would be no central body to ensure certain rules, people would do what they want, as they will and life would be "nasty, brutish and short" \cite{hobbes} \cite[Ch.\ 8]{leviathan} \cite{hobbespod}. Hence, a central government is required so as to ensure that society doesn't dive into chaos. Hobbes argued that since such an authority will ensure the safety and stability of people \cite{philosophy}, they should obey the authority, with only a few rights to complain, and those too when they are directly threatened by the authority. 
While Hobbes' arguments made sense at that time for some people because the English Civil War had just shown them what the "state of nature" was like, it was not a very efficient method of governance, since it promoted totalitarian governments like direct monarchy or dictatorship and a rule of tyrants, where a minority of people have all the power over others and may do as they wish.

Whereas Hobbes' social contract theory argues for ultimate power to the sovereign, John Locke vouched for limited power to the sovereign in return for the protection of natural rights of the people like liberty and justice \cite{philosophy}. Locke put forward his revolutionary ideas through his 1689 book \textbf{Two Treatises of Government}, which tried to answer the question of who should rule the country and on what basis. He took on Hobbes' idea of the state of nature by arguing that a state of nature indeed existed, but instead of being a chaotic mess where people are just after each others' lives, it would be broadly peaceful, and people would form a central government just to ensure that no one has a right to cause harm to others. People would not surrender all of their rights; they would possess some rights that the sovereign could never take away. If they feel like the sovereign is acting like a tyrant or is not preserving their basic rights, they can overthrow the ruler. \cite[Ch.\ 14]{leviathan} \cite{locke}.
Locke's ideas played a huge role in the development of western societies, especially in the American Declaration of Independence.

Jean-Jacques Rousseau developed a new social contract theory that went even further than Locke's theory, that balanced the freedom of the individual with the power of the sovereign \cite{philosophy}.
He argued that the people should become the sovereign. People should themselves have the power to enforce their rights, and these powers should not lie in the hands of a minority of people. Rousseau also was a believer of "romanticism", where he argued that modern technology and sciences have led to people being plagued by vice and sin. His idea of a "state of nature" was that of a world where people led simple, peaceful, and satisfied lives, with the love of family, respect for nature, awe at the beauty of the universe, curiosity about others, empathy, kindness and a taste for music and simple entertainments being the sole principles \cite{rousseau}. He argued that the move to civilizations led to people losing their self-identities to those of others.
\\

While all three of the social contract theories espoused by Hobbes, Locke, and Rousseau followed conflicting views and varied based on their view on human nature and politics, all of them vouched for one basic principle: the society should be free to exercise their rights.
\\

Out of these three social contract theories, the theory which bears considerable resemblance to the present State of India is John Locke's social contract theory where limited power is given to the central government. The Indian State follows a democratic form of government, where the central authorities are selected by the people as a whole, and they have limited power and responsibility to ensure the safety and well-being of the society and to ensure justice among the people. This is quite close to Locke's idea of a society where people recognize each other's rights and people can overthrow the running government if they feel like it is not valuing the power given to it, or is misusing the power given to it.

\bibliographystyle{unsrt}
\bibliography{eco}

\end{document}